% This is LLNCS.DEM the demonstration file of
% the LaTeX macro package from Springer-Verlag
% for Lecture Notes in Computer Science,
% version 2.3 for LaTeX2e
%
\documentclass{llncs}
%
\usepackage{makeidx}  % allows for indexgeneration
\usepackage[utf8]{inputenc}
\usepackage[english]{babel}
\usepackage{indentfirst}
%
\begin{document}

\mainmatter              % start of the contributions
%
\title{Text Mining Wikipedia\\to extract historical facts}
%
\titlerunning{Hamiltonian Mechanics}  % abbreviated title (for running head)
%                                     also used for the TOC unless
%                                     \toctitle is used
%
\author{João Valente \and João Gradim}
%
\authorrunning{João Valente, João Gradim}   % abbreviated author list (for running head)
%
%%%% list of authors for the TOC (use if author list has to be modified)
\tocauthor{João Valente, João Gradim}
%
\institute{Faculdade de Engenharia da Universidade do Porto,\\
Rua do Dr. Roberto Frias, s/n, Porto, Portugal}

\maketitle              % typeset the title of the contribution

\begin{abstract}
The abstract should summarize the contents of the paper
using at least 70 and at most 150 words. It will be set in 9-point
font size and be inset 1.0 cm from the right and left margins.
There will be two blank lines before and after the Abstract. \dots
\end{abstract}

%
\section{The problem}

Information relative to historical events is readily available on Wikipedia. However, this information is not easily searchable: it requires manual searching through entire articles to find a specific piece of information.

%
\section{Objectives}

The main objectives of this project are:

\begin{itemize}
	\item to provide an easily queryable database of historical events of major importance
	\item To allow users to use natural language to perform queries
	\item To be able to cross-reference historical events and link figures, places
\end{itemize}

\section{Motivation}

Younger population knows progressively less and less about history and human achievements. A simple interface would provide a means to an easy access to information and could boost interest in learning.\\

\section{State of the Art}

\subsection{Natural Language Processing and Tokenization}

\begin{itemize}
	\item The Stanford Parser
\end{itemize}

\subsection{HTML Parsing (Ruby)}

\begin{itemize}
	\item HPricot
	\item Nokogiri
\end{itemize}

\subsection{Classification and  Machine Learning}

\begin{itemize}
	\item Support Vector Machines
	\item Naïve Bayes Classification
	\begin{itemize}
		\item Bayesian Networks
	\end{itemize}
\end{itemize}

\section{Current Work}

\begin{itemize}
	\item Research
	\item Text structure and markup analysis
	\item Tool testing
\end{itemize}

%
% ---- Bibliography ----
%
\begin{thebibliography}{}
%
\bibitem[1980]{2clar:eke}
Clarke, F., Ekeland, I.:
Nonlinear oscillations and
boundary-value problems for Hamiltonian systems.
Arch. Rat. Mech. Anal. 78, 315--333 (1982)

\end{thebibliography}
\clearpage
\addtocmark[2]{Author Index} % additional numbered TOC entry
\renewcommand{\indexname}{Author Index}
\printindex
\clearpage
\addtocmark[2]{Subject Index} % additional numbered TOC entry
\markboth{Subject Index}{Subject Index}
\renewcommand{\indexname}{Subject Index}
\input{subjidx.ind}
\end{document}
